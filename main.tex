\documentclass[10pt,a4paper]{article}
\usepackage[utf8]{inputenc}

\usepackage[landscape,margin=1cm]{geometry}
\usepackage[english]{babel}


% colour themes to come. KnitR?

%-------------------------

\title{\color{w3schools}Colorful {\color{alert} Cheatsheet}: {\color{black} A Template}}
\author{\LaTeX{} Ninja}
\date{July 2019}

\usepackage[default]{raleway}
\usepackage{fontawesome}
\usepackage[T1]{fontenc}

\usepackage{minted}  
\usemintedstyle{friendly}  

\usepackage{hyperref}
\usepackage{enumitem}
\usepackage{lipsum}

\usepackage{xcolor}
\definecolor{customcolor}{HTML}{616AC5}
\definecolor{alert}{HTML}{CD5C5C}
\definecolor{w3schools}{HTML}{4CAF50}
\definecolor{subbox}{gray}{0.60}
\definecolor{codecolor}{HTML}{FFC300}
\definecolor{whitebox}{rgb}{1,1,1} % RGB white
\colorlet{xx}{customcolor}


%--------------------------Editor mode.

\usepackage
[citestyle=authoryear,
sorting=nty,	  		%Sorts bibliography by year, name, title
autocite=footnote, 		%Autocite command generates footnotes
autolang=hyphen, 		
mincrossrefs=1, 	
backend=biber]
{biblatex}

\DeclareFieldFormat{postnote}{#1}
\DeclareFieldFormat{multipostnote}{#1}
\DeclareAutoCiteCommand{footnote}[f]{\footcite}{\footcites}

\bibliography{literature}
%----------------------------------------
%--------------------------------------------------------------------------------
\usepackage{tcolorbox}

\tcbuselibrary{most,listingsutf8,minted}

\tcbset{tcbox width=auto,left=1mm,top=1mm,bottom=1mm,
right=1mm,boxsep=1mm,middle=1pt}


\newenvironment{mycolorbox}[2]{%
\begin{tcolorbox}[grow to left by=-1em,grow to right by=-1em,capture=minipage,fonttitle=\large\bfseries, enhanced jigsaw,boxsep=1mm,colback=#1!30!white,on line,tcbox width=auto, toptitle=0mm,colframe=#1,opacityback=0.7,nobeforeafter,title=#2]%
}{\end{tcolorbox}\\[0.2em]}

\newenvironment{subbox}[2]{%
\begin{tcolorbox}[capture=minipage,fonttitle=\normalsize\bfseries, enhanced jigsaw,boxsep=1mm,colback=#1!30!white,on line,tcbox width=auto,left=0.3em,top=1mm, toptitle=0mm,colframe=#1,opacityback=0.7,nobeforeafter,title=#2]\footnotesize %
}{\normalsize\end{tcolorbox}\vspace{0.1em}}

\newenvironment{multibox}[1]{%
\begin{tcbraster}[raster columns=#1,raster equal height,nobeforeafter,raster column skip=1em,raster left skip=1em,raster right skip=1em]}{\end{tcbraster}}

\newenvironment{textbox}[1]{\begin{mycolorbox}{customcolor}{#1}}{\end{mycolorbox}}

% Define more colored boxes like textbox, but with different colors
\newenvironment{textboxGreen}[1]{\begin{mycolorbox}{w3schools}{#1}}{\end{mycolorbox}}
\newenvironment{textboxRed}[1]{\begin{mycolorbox}{alert}{#1}}{\end{mycolorbox}}
\newenvironment{textboxPurple}[1]{\begin{mycolorbox}{customcolor}{#1}}{\end{mycolorbox}}
\newenvironment{textboxGray}[1]{\begin{mycolorbox}{subbox}{#1}}{\end{mycolorbox}}
\newenvironment{textboxYellow}[1]{\begin{mycolorbox}{codecolor}{#1}}{\end{mycolorbox}}
\tcbset{
  whitebox/.style={
    colback=white,       % background color
    colframe=white,      % border color
    coltitle=black,      % title text color (if any)
    boxrule=0pt,         % no border line thickness
    left=0pt, right=0pt, top=0pt, bottom=0pt, % no padding (optional)
    enhanced,
    sharp corners         % no rounded corners
  }
}
\newenvironment{textboxWhite}[1]{\begin{mycolorbox}{whitebox}{#1}}{\end{mycolorbox}}
%-------------------------------
\newtcblisting{codebox}[2]{colback=codecolor!5,colframe=codecolor!80!black,listing only, 
minted options={numbers=left,style=friendly,fontsize=\tiny,breaklines,autogobble,linenos,numbersep=3mm},
left=5mm,enhanced,
title=#2, fonttitle=\bfseries,
listing engine=minted,minted language=#1}

%--------------------------------------------------------------------------------
\newcommand{\punkti}{~\lbrack\dots\rbrack~}

\renewenvironment{quote}
               {\list{\faQuoteLeft\phantom{ }}{\rightmargin\leftmargin}%
                \item\relax\scriptsize\ignorespaces}
               {\unskip\unskip\phantom{xx}\faQuoteRight\endlist}
               

%--------------------------------------------------------------------------------
\newcommand{\bgupper}[3]{\colorbox{#1}{\color{#2}\huge\bfseries\MakeUppercase{#3}}}
\newcommand{\bg}[3]{\colorbox{#1}{\bfseries\color{#2}#3}}

\newcommand{\mycommand}[2]{{\ttfamily\detokenize{#1}}~\dotfill{}~{\footnotesize #2}\\}
\newcommand{\sep}{{\scriptsize~\faCircle{ }~}}


\newcommand{\bggreen}[1]{\medskip\bgupper{w3schools}{black}{#1}\\[0.5em]}
\newcommand{\green}[1]{\smallskip\bg{w3schools}{white}{#1}\\}
\newcommand{\red}[1]{\smallskip\bg{alert}{white}{#1}\\}

\usepackage{multicol}
\setlength{\columnsep}{30pt}

\setlength{\parindent}{0pt}
\pagestyle{empty}

\usepackage{csquotes}

\newcommand{\loremipsum}{Lorem ipsum dolor sit amet.}




%--------------------------------------------------------------------------------
\begin{document}
\small
\begin{multicols}{3}

\maketitle
\thispagestyle{empty}
\scriptsize

\section{Overview}


\begin{textbox}{Lorem Ipsum}


\begin{quote}
    \loremipsum \textbf{\loremipsum} \loremipsum \loremipsum \loremipsum \loremipsum
\end{quote}

\begin{quote}
    ``Parents of young organic life forms should be warned, that
towels can be harmful, if swallowed in large quantities.''
\end{quote}

\begin{quote}
        \emph{A fool with a tool is still a fool!} (\href{https://en.wikiquote.org/wiki/Talk:Grady_Booch}{Grady Booch})
    \end{quote}

\end{textbox}


\begin{textbox}{Key concepts I}
\red{Bag-of-words (BOW)}

\green{\emph{type} vs. \emph{token}}
\emph{To be or not to be.} = 6 \emph{tokens}, 4 \emph{types}.

\red{\emph{type}} descriptive criterion\footcite[cf.][12]{stroustrup}

\red{\emph{token}} unit of analysis\footcites[cf.][12]{stroustrup,attentionMerchants}


\bigskip

\underline{Key topics}
\begin{itemize}
    \item One
    \item Two
    \item Three
\end{itemize}

\end{textbox}









%--------------------------------------------------------------
\section{Tools}



\begin{textbox}{Lorem Ipsum}
test  \sep test \sep test \sep test

\bigskip

\green{Visualize}
\begin{enumerate}
\item this is a \textbf {big} testing data for a cheatsheet 
\item it has \textbf {numbers} to better represent the information
\item again, it has \textbf {numbers} to better represent the information
\end{enumerate}

\end{textbox}


\begin{textbox}{Voyant Tools}
 \href{https://voyant-tools.org/}{Voyant Tools} \sep \href{https://voyant-tools.org/}{Voyant Tools}

\includegraphics[width=\textwidth]{voyant.png}
\end{textbox}




\begin{textbox}{Project Gutenberg Texts}
\begin{tabular}{r|p{0.8\textwidth}}\scriptsize
    84 & \href{http://www.gutenberg.org/ebooks/84}{Frankenstein; Or, The Modern Prometheus by Mary Wollstonecraft Shelley} \\
    6087 & \href{https://www.gutenberg.org/ebooks/6087}{The Vampyre; a Tale by John William Polidori} \\
    696 & \href{https://www.gutenberg.org/ebooks/696}{The Castle of Otranto by Horace Walpole} \\
    42 & \href{https://www.gutenberg.org/ebooks/42}{The Strange Case of Dr. Jekyll and Mr. Hyde by Robert Louis Stevenson}
\end{tabular}

\end{textbox}





\begin{textbox}{Key concepts}
\red{Bag-of-words (BOW)}


\green{\emph{Zipf's Law}}

\mycommand{_&§!$§/()$}{code}
\mycommand{shutdown -h now}{to shutdown}

\end{textbox}


%--------------------------------------------------------------
\section{Programming}

\subsection{Code boxes}


% first argument: minted programming language name, for example.. css, c, cpp, etc.
\begin{codebox}{r}{Code box using R}
# Install
install.packages("tm")  # for text mining

# Load
library("tm")

# text <- readLines(file.choose())
# Read the text file from internet
filePath <- "http://www.internet.com/text.txt"
text <- readLines(filePath)

\end{codebox}


% first argument: minted programming language name, for example.. css, c, cpp, etc.
\begin{codebox}{cpp}{Code box using C++}
for (auto element : vector) 
{
    sum += element;
}
\end{codebox}

\end{multicols}

\end{document}
