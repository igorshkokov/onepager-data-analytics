\documentclass[10pt,a4paper]{article}
\usepackage[utf8]{inputenc}

\usepackage[landscape,margin=1cm]{geometry}
\usepackage[english]{babel}

\date{} % disables the default date
\title{How to Get a Job as a \textbf{Data Analyst}}
\author{Igor Shkokov}

\usepackage[default]{raleway}
\usepackage{fontawesome}
\usepackage[T1]{fontenc}

\usepackage{minted}  
\usemintedstyle{friendly}  

\usepackage{hyperref}
\usepackage{enumitem}
\usepackage{lipsum}

\usepackage{xcolor}
\definecolor{customcolor}{HTML}{616AC5}
\definecolor{alert}{HTML}{CD5C5C}
\definecolor{w3schools}{HTML}{4CAF50}
\definecolor{subbox}{gray}{0.60}
\definecolor{codecolor}{HTML}{FFC300}
\definecolor{whitebox}{rgb}{1,1,1} % RGB white
\colorlet{xx}{customcolor}


%--------------------------Editor mode.

\usepackage
[citestyle=authoryear,
sorting=nty,	  		%Sorts bibliography by year, name, title
autocite=footnote, 		%Autocite command generates footnotes
autolang=hyphen, 		
mincrossrefs=1, 	
backend=biber]
{biblatex}

\DeclareFieldFormat{postnote}{#1}
\DeclareFieldFormat{multipostnote}{#1}
\DeclareAutoCiteCommand{footnote}[f]{\footcite}{\footcites}

\bibliography{literature}
%----------------------------------------
%--------------------------------------------------------------------------------
\usepackage{tcolorbox}

\tcbuselibrary{most,listingsutf8,minted}

\tcbset{tcbox width=auto,left=1mm,top=1mm,bottom=1mm,
right=1mm,boxsep=1mm,middle=1pt}


\newenvironment{mycolorbox}[2]{%
\begin{tcolorbox}[grow to left by=-1em,grow to right by=-1em,capture=minipage,fonttitle=\large\bfseries, enhanced jigsaw,boxsep=1mm,colback=#1!30!white,on line,tcbox width=auto, toptitle=0mm,colframe=#1,opacityback=0.7,nobeforeafter,title=#2]%
}{\end{tcolorbox}\\[0.2em]}

\newenvironment{subbox}[2]{%
\begin{tcolorbox}[capture=minipage,fonttitle=\normalsize\bfseries, enhanced jigsaw,boxsep=1mm,colback=#1!30!white,on line,tcbox width=auto,left=0.3em,top=1mm, toptitle=0mm,colframe=#1,opacityback=0.7,nobeforeafter,title=#2]\footnotesize %
}{\normalsize\end{tcolorbox}\vspace{0.1em}}

\newenvironment{multibox}[1]{%
\begin{tcbraster}[raster columns=#1,raster equal height,nobeforeafter,raster column skip=1em,raster left skip=1em,raster right skip=1em]}{\end{tcbraster}}

\newenvironment{textbox}[1]{\begin{mycolorbox}{customcolor}{#1}}{\end{mycolorbox}}

% Define more colored boxes like textbox, but with different colors
\newenvironment{textboxGreen}[1]{\begin{mycolorbox}{w3schools}{#1}}{\end{mycolorbox}}
\newenvironment{textboxRed}[1]{\begin{mycolorbox}{alert}{#1}}{\end{mycolorbox}}
\newenvironment{textboxPurple}[1]{\begin{mycolorbox}{customcolor}{#1}}{\end{mycolorbox}}
\newenvironment{textboxGray}[1]{\begin{mycolorbox}{subbox}{#1}}{\end{mycolorbox}}
\newenvironment{textboxYellow}[1]{\begin{mycolorbox}{codecolor}{#1}}{\end{mycolorbox}}
\tcbset{
  whitebox/.style={
    colback=white,       % background color
    colframe=white,      % border color
    coltitle=black,      % title text color (if any)
    boxrule=0pt,         % no border line thickness
    left=0pt, right=0pt, top=0pt, bottom=0pt, % no padding (optional)
    enhanced,
    sharp corners         % no rounded corners
  }
}
\newenvironment{textboxWhite}[1]{\begin{mycolorbox}{whitebox}{#1}}{\end{mycolorbox}}
%-------------------------------
\newtcblisting{codebox}[2]{colback=codecolor!5,colframe=codecolor!80!black,listing only, 
minted options={numbers=left,style=friendly,fontsize=\tiny,breaklines,autogobble,linenos,numbersep=3mm},
left=5mm,enhanced,
title=#2, fonttitle=\bfseries,
listing engine=minted,minted language=#1}

%--------------------------------------------------------------------------------
\newcommand{\punkti}{~\lbrack\dots\rbrack~}

\renewenvironment{quote}
               {\list{\faQuoteLeft\phantom{ }}{\rightmargin\leftmargin}%
                \item\relax\scriptsize\ignorespaces}
               {\unskip\unskip\phantom{xx}\faQuoteRight\endlist}
               

%--------------------------------------------------------------------------------
\newcommand{\bgupper}[3]{\colorbox{#1}{\color{#2}\huge\bfseries\MakeUppercase{#3}}}
\newcommand{\bg}[3]{\colorbox{#1}{\bfseries\color{#2}#3}}

\newcommand{\mycommand}[2]{{\ttfamily\detokenize{#1}}~\dotfill{}~{\footnotesize #2}\\}
\newcommand{\sep}{{\scriptsize~\faCircle{ }~}}


\newcommand{\bggreen}[1]{\medskip\bgupper{w3schools}{black}{#1}\\[0.5em]}
\newcommand{\green}[1]{\smallskip\bg{w3schools}{white}{#1}\\}
\newcommand{\red}[1]{\smallskip\bg{alert}{white}{#1}\\}

\usepackage{multicol}
\setlength{\columnsep}{30pt}

\setlength{\parindent}{0pt}
\pagestyle{empty}

\usepackage{csquotes}

\newcommand{\loremipsum}{Lorem ipsum dolor sit amet.}


\begin{document}
\small
\begin{multicols}{3}

\maketitle
\thispagestyle{empty}
\scriptsize
%-----------------------------------------------------

\begin{textbox}{Applying} 

\begin{itemize}                                                    
    \item \emph{It's a game of stakes - you need to be lucky, but also collect this luck. Everyone gets through this at some point in their life - you will do too.}
    \item \textbf{Apply often} (15–20 roles/week)
    \item Tailor your CV for each application \textbf{(1 custom > 5 generic)}
    \item Most companies use Applicant Tracking Systems (ATS), so your real CV is never seen by an HR. Use \textbf{keywords} Tools: Jobscan
    \item Try to find HR or team manager online and contact them in DM
    \item 99\% rejection rate is normal — always ask for feedback, even after an automated mail
    \item Ask friends in HR to review your CV
    \item Don't worry about having a polished GitHub - no one expects a junior Data Analyst to have 100 star repositories
\end{itemize}

\end{textbox}	
%-----------------------------------------------------

\begin{textboxGray}{If You're Not Getting Interviews}
\begin{itemize}
    \item \textbf{Make it impossible to ignore your application} via:
    \begin{itemize}
        \item Pet projects \textbf{that stand out}
        \item Referrals (even asking a stranger on Linkedin to refer you may work)
        \item Domain expertise
    \end{itemize}
    \item If your success rate is below 1-2\% (less than 3 out of 100 of your applications are getting a reply) - change your approach (work on your CV, advance on pet projects, etc)
\end{itemize}
\end{textboxGray}
%-----------------------------------------------------

\begin{textboxRed}{Creating Memorable Pet Projects}

\begin{itemize}
    \item  \emph{A good pet project that really gets the attention of a person in a company may compensate the lack of experience. I got my first job in Data by sending this (link) youtube video with every application.}
    \item Ideal pet project = intersection of (stack of the role you're applying to+a topic you care about (sports, finance, games, etc.)
    \item Find unexpected results
    \item Show the process:
    \begin{enumerate}
        \item Define Question
        \item Data Acquisition (API, scraping, Kaggle, gov data)
        \item Cleaning and Wrangling
        \item Analysis (SQL, Python)
        \item Visualization (BI tools, Python)
        \item Insights and Recommendations
    \end{enumerate}
    \item Make it \textbf{showable} and \textbf{presentable} (YouTube > all)
    \item Examples:
    \begin{itemize}
        \item My YouTube on Paris
        \item Another YouTube Presentation
        \item Tableau Public dashboard
        \item Medium post
        \item One-page site (GitHub Pages, Tableau)
    \end{itemize}
\end{itemize}
Learning and presenting is easier with outcome-driven projects.
\end{textboxRed}
%-----------------------------------------------------

\begin{textbox}{Focus Areas}
\begin{tabular}{l|l|l}
\textbf{Skill} & \textbf{Focus \%} & \textbf{Examples} \\
\hline
SQL & 50\% & Joins, Window functions, CTEs \\
DataViz & 30\% & Tableau, Power BI, Looker \\
Python & 10\% & Pandas, NumPy \\
Statistics & 5\% & A/B testing, distributions, causality \\
Other Skills & 3\% & Excel, Airflow, dbt \\
Cloud & 2\% & GCP, AWS, Snowflake \\
\end{tabular}
\end{textbox}
%-----------------------------------------------------
\begin{textboxGreen}{If You Get an Interview}
Well done! Here are the usual steps - some of them may be skipped or reordered, but usually:
   \begin{enumerate}
        \item HR call
        \item Technical test
        \item Presenting test
        \item Another tech interview (live coding,etc)
        \item Team fit interview
        \item Offer
    \end{enumerate}
Each step could be the last one - there are multiple persons competing for the same position.
\begin{itemize}
    \item Recruiters often just filter — don't expect too much
    \item If you get to a technical test, spend as much time on it as you can
    \item Smart questions are sometimes more important than smart answers - ask smart questions:
    \begin{itemize}
        \item e.g. \emph{What data challenges does your team face?}
        \item e.g. \emph{Isn't [this particular tech from their stack] outdated? Did you think about [this new tech that's more innovative]?}
    \end{itemize}
    \item On each step, ask how many other candidates do they see on this step. Even if they won't answer, you'll get an idea.
    \item Put yourself in shoes of your interviewers. What are their pains? What are they trying to optimize for?
\end{itemize}
\end{textboxGreen}


%-----------------------------------------------------
\begin{textboxGray}{Courses and Practice}
Practice often, especially before a live coding test.
\begin{itemize}
    \item  \href{https://datacamp.com/}{DataCamp} (SQL, Python) — best overall
    \item SQL Trainings:
    \begin{itemize}
        \item DataLemur (link)
        \item SQL Noir (link)
        \item \href{https://mystery.knightlab.com/}{SQL Mystery}
    \end{itemize}
    \item Public datasets: \textbf{Kaggle}
    \item Coding challenges: Leetcode, Stratascratch
\end{itemize}
\end{textboxGray}
%-----------------------------------------------------
\begin{textboxYellow}{Reading list}
\textbf{Blogs:}
\begin{itemize}
    \item Towards Data Science
    \item Kdnuggets
    \item \href{https://ourworldindata.org/}{Our World in Data}
    \item Voronoi
    \item Reddit: \texttt{r/dataanalysis}
\end{itemize}
\textbf{Newsletters:} TLDR, FlowingData, blef.fr, HolonIQ

\textbf{Books:}
\begin{itemize}
    \item Designing Your Life (Part 7)
    \item Wishcraft (Part 4)
\end{itemize}
\end{textboxYellow}
%-----------------------------------------------------

\begin{textbox}{One Last Thought}
\emph{Would you do this if you had unlimited money?}

If your answer is no — maybe reconsider.
\end{textbox}
\date{\today}
\end{multicols}
\end{document}
